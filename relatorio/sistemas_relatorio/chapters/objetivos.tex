\par O propósito desse trabalho é desenvolver um sistema capaz de controlar o
protótipo do modelo real de um elevador de 4 andares utilizando um chip FPGA
para processar informações digitais e enviar sinais de comando para um motor DC
-- ou de corrente contínua, componente responsável pelo movimento da cabine.
\par Esse trabalho também aborda detalhes sobre a aplicação da linguagem de
descrição de hardware Verilog no design de sistemas afins e apresenta onde essa 
ferramenta diferencia-se das linguagens de programação mais comuns da indústria.
\par O FPGA é um chip desenhado para ser configurado pelo projetista, por isso o
nome \textit{Field-Programmable}, ou programável em campo, essa característica
possibilita ao designer projetar, verificar e implementar sistemas 
\textit{on-the-fly}, rapidamente, garantindo a versatilidade e operabilidade
imediata de sistemas complexos. Há nesse trabalho a verificação do potencial desse 
dispositivo para desenvolvimento estratégico e rápido de sistemas digitais de 
controle, dessa maneira, justificando seu uso extensivo em diversas aplicações na 
indústria.
\par Ao longo desse relatório, serão descritas todas as etapas do projeto, desde a 
escolha do design de controle ao desenvolvimento do modelo físico do elevador
seguindo o seguinte esquema lógico:
\vfill
\emph{Estrutura do relatório -- esquema lógico}\vspace{2pt}
\normalsize
\hrule
\begin{center}
	\begin{itemize}
		\item \emph{O Elevador}
		\begin{itemize} \renewcommand{\labelitemi}{$\Rightarrow$}
			\item As características da estrutura física
			\item Os elementos do elevador
			\item Sensores
			\item A interface de entrada
			\item O projeto elétrico
		\end{itemize}
		\item \emph{O Controle}
		\begin{itemize} \renewcommand{\labelitemi}{$\Rightarrow$}
			\item As variáveis de controle 
			\item O circuito de controle 
			\item A máquina de estados finita 
		\end{itemize}
	\end{itemize}
\end{center}
\hrule
